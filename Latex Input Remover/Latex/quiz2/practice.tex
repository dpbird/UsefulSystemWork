\documentclass[12pt,addpoints]{exam}

\newcommand{\class}{10-423/10-623 Gen AI}
\newcommand{\term}{Fall 2024}
\newcommand{\examnum}{Quiz 2}
\newcommand{\examdate}{09/25/24}
\newcommand{\timelimit}{15 minutes} % This one was 18-20 minutes in S24
%% To HIDE SOLUTIONS, set this value to 0: 
%\providecommand{\issoln}{0}
\providecommand{\issoln}{1}

\usepackage{amsmath, amssymb, amsthm, enumerate, graphicx}
\usepackage[usenames,dvipsnames]{color}
\usepackage{todonotes}
\usepackage{bm}
\usepackage[colorlinks=true,urlcolor=blue]{hyperref}
\usepackage{geometry}
\geometry{margin=1in}
\usepackage{float}
\usepackage{graphics}
\setlength{\marginparwidth}{2.15cm}
\usepackage{booktabs}
\usepackage{enumitem}
\usepackage{epsfig}
\usepackage{setspace}
\usepackage{parskip}
\usepackage[normalem]{ulem}
\usepackage{tikz}
\usetikzlibrary{positioning, arrows, automata}
\usepackage{pgfplots}
\usepgfplotslibrary{fillbetween}
\usepackage[font=scriptsize]{subcaption}
\usepackage{float}
\usepackage{algorithmicx}
\usepackage[noend]{algpseudocode}
\usepackage{environ}
\usepackage{bbm}
\usepackage{graphicx}
\usepackage{titling}
\usepackage{url}
\usepackage{xcolor}
\usepackage{lipsum}
\usepackage{lastpage}
\usepackage[colorlinks=true,urlcolor=blue]{hyperref}
\usepackage{multicol}
\usepackage{tabularx}
\usepackage{comment}
\usepackage{amsmath}
\usepackage{nicefrac}
\usepackage[tableposition=top]{caption}
\usepackage[many]{tcolorbox}
\usepackage{colortbl}
\usepackage{array}
\usepackage{multirow}
\usepackage{listings}
\usepackage{color}
\usepackage{adjustbox}
\usepackage{wasysym} % For \CIRCLE
\usepackage{cancel} % For \xcancel

\pgfplotsset{compat=1.16}


\definecolor{dkgreen}{rgb}{0,0.6,0}
\definecolor{gray}{rgb}{0.5,0.5,0.5}
\definecolor{mauve}{rgb}{0.58,0,0.82}

\lstset{frame=tb,
  language=Python,
  aboveskip=3mm,
  belowskip=3mm,
  showstringspaces=false,
  columns=flexible,
  basicstyle={\small\ttfamily},
  numbers=none,
  numberstyle=\tiny\color{gray},
  keywordstyle=\color{blue},
  commentstyle=\color{dkgreen},
  stringstyle=\color{mauve},
  breaklines=true,
  breakatwhitespace=true,
  tabsize=3
}


\newcommand{\class}{10-301/601 Machine Learning}
\newcommand{\term}{Spring 2023}
\newcommand{\examnum}{Exam 2}
\newcommand{\examdate}{11/09/2023}
\newcommand{\timelimit}{120 minutes}
\newcommand{\argmax}{\operatornamewithlimits{arg\,max}}
\newcommand{\argmin}{\operatornamewithlimits{arg\,min}}

% Instead of lines, use blank space.
%\renewcommand{\fillwithlines}[1]{\vspace{#1}}

\def\x{\mathbf x}
\def\y{\mathbf y}
\def\w{\mathbf w}
\def\v{\mathbf v}
\def\E{\mathbb E}
\def\V{\mathbb V}
\def\a{\mathbf a}
\def\z{\mathbf z}

\newcommand\MyBox[1]{%
  \fbox{\parbox[c][1.7cm][c]{1.7cm}{\centering #1}}%
}
\newcommand\MyVBox[1]{%
  \parbox[c][1.7cm][c]{2.5cm}{\centering\bfseries #1}%
}  
\newcommand\MyHBox[2][\dimexpr1.7cm+2\fboxsep\relax]{%
  \parbox[c][1cm][c]{#1}{\centering\bfseries #2}%
}  
\newcommand\MyTBox[3]{
  \MyVBox{#1}\MyBox{#2}
  \MyBox{#3}\par
}


\newcommand{\pts}[1]{(#1 points)}

% SOLUTION environment
\newenvironment{soln}{\leavevmode\color{red}\ignorespaces }{}

% QUESTION AUTHORS environment
\newenvironment{qauthor}{\leavevmode\color{blue}\ignorespaces }{}

% Question tester comment environment
\newenvironment{qtester}{\leavevmode\color{green}\ignorespaces}{}

% TO ONLY SHOW HOMEWORK QUESTIONS, include following (else comment out):
   % \RenewEnviron{soln}{}
   \RenewEnviron{qauthor}{}
  \RenewEnviron{qtester}{}

\newcommand{\norm}[1]{\lVert #1 \rVert}
\newcommand{\st}{\mathrm{s.t.}}

\makeatletter
\newcommand{\removelatexerror}{\let\@latex@error\@gobble}
\makeatother

\setlength\linefillheight{.35in}

%%%%%%%%%%%%%%%%%%%%%%%%%%%%%%%%%%%%%%%%%%
% Custom commands                        %
%%%%%%%%%%%%%%%%%%%%%%%%%%%%%%%%%%%%%%%%%%

% First argument is width, second argument is label.
\newcommand{\blankforFITB}[2]{\underline{\hspace{#1}#2\hspace{#1}}}

\newcommand{\vc}[1]{\boldsymbol{#1}}

\newcommand{\fpartial}[2]{\frac{\partial #1}{\partial #2}}
\newcommand{\adj}[1]{\frac{\partial J}{\partial #1}}
\newcommand{\chain}[2]{\adj{#2} = \adj{#1}\frac{\partial #1}{\partial #2}}

% mathcal
\newcommand{\Ac}{\mathcal{A}}
\newcommand{\Bc}{\mathcal{B}}
\newcommand{\Cc}{\mathcal{C}}
\newcommand{\Dc}{\mathcal{D}}
\newcommand{\Ec}{\mathcal{E}}
\newcommand{\Fc}{\mathcal{F}}
\newcommand{\Gc}{\mathcal{G}}
\newcommand{\Hc}{\mathcal{H}}
\newcommand{\Ic}{\mathcal{I}}
\newcommand{\Jc}{\mathcal{J}}
\newcommand{\Kc}{\mathcal{K}}
\newcommand{\Lc}{\mathcal{L}}
\newcommand{\Mc}{\mathcal{M}}
\newcommand{\Nc}{\mathcal{N}}
\newcommand{\Oc}{\mathcal{O}}
\newcommand{\Pc}{\mathcal{P}}
\newcommand{\Qc}{\mathcal{Q}}
\newcommand{\Rc}{\mathcal{R}}
\newcommand{\Sc}{\mathcal{S}}
\newcommand{\Tc}{\mathcal{T}}
\newcommand{\Uc}{\mathcal{U}}
\newcommand{\Vc}{\mathcal{V}}
\newcommand{\Wc}{\mathcal{W}}
\newcommand{\Xc}{\mathcal{X}}
\newcommand{\Yc}{\mathcal{Y}}
\newcommand{\Zc}{\mathcal{Z}}

% mathbb
\newcommand{\Ab}{\mathbb{A}}
\newcommand{\Bb}{\mathbb{B}}
\newcommand{\Cb}{\mathbb{C}}
\newcommand{\Db}{\mathbb{D}}
\newcommand{\Eb}{\mathbb{E}}
\newcommand{\Fb}{\mathbb{F}}
\newcommand{\Gb}{\mathbb{G}}
\newcommand{\Hb}{\mathbb{H}}
\newcommand{\Ib}{\mathbb{I}}
\newcommand{\Jb}{\mathbb{J}}
\newcommand{\Kb}{\mathbb{K}}
\newcommand{\Lb}{\mathbb{L}}
\newcommand{\Mb}{\mathbb{M}}
\newcommand{\Nb}{\mathbb{N}}
\newcommand{\Ob}{\mathbb{O}}
\newcommand{\Pb}{\mathbb{P}}
\newcommand{\Qb}{\mathbb{Q}}
\newcommand{\Rb}{\mathbb{R}}
\newcommand{\Sb}{\mathbb{S}}
\newcommand{\Tb}{\mathbb{T}}
\newcommand{\Ub}{\mathbb{U}}
\newcommand{\Vb}{\mathbb{V}}
\newcommand{\Wb}{\mathbb{W}}
\newcommand{\Xb}{\mathbb{X}}
\newcommand{\Yb}{\mathbb{Y}}
\newcommand{\Zb}{\mathbb{Z}}

% mathbf lowercase
\newcommand{\av}{\mathbf{a}}
\newcommand{\bv}{\mathbf{b}}
\newcommand{\cv}{\mathbf{c}}
\newcommand{\dv}{\mathbf{d}}
\newcommand{\ev}{\mathbf{e}}
\newcommand{\fv}{\mathbf{f}}
\newcommand{\gv}{\mathbf{g}}
\newcommand{\hv}{\mathbf{h}}
\newcommand{\iv}{\mathbf{i}}
\newcommand{\jv}{\mathbf{j}}
\newcommand{\kv}{\mathbf{k}}
\newcommand{\lv}{\mathbf{l}}
\newcommand{\mv}{\mathbf{m}}
\newcommand{\nv}{\mathbf{n}}
\newcommand{\ov}{\mathbf{o}}
\newcommand{\pv}{\mathbf{p}}
\newcommand{\qv}{\mathbf{q}}
\newcommand{\rv}{\mathbf{r}}
\newcommand{\sv}{\mathbf{s}}
\newcommand{\tv}{\mathbf{t}}
\newcommand{\uv}{\mathbf{u}}
\newcommand{\vv}{\mathbf{v}}
\newcommand{\wv}{\mathbf{w}}
\newcommand{\xv}{\mathbf{x}}
\newcommand{\yv}{\mathbf{y}}
\newcommand{\zv}{\mathbf{z}}

% mathbf uppercase
\newcommand{\Av}{\mathbf{A}}
\newcommand{\Bv}{\mathbf{B}}
\newcommand{\Cv}{\mathbf{C}}
\newcommand{\Dv}{\mathbf{D}}
\newcommand{\Ev}{\mathbf{E}}
\newcommand{\Fv}{\mathbf{F}}
\newcommand{\Gv}{\mathbf{G}}
\newcommand{\Hv}{\mathbf{H}}
\newcommand{\Iv}{\mathbf{I}}
\newcommand{\Jv}{\mathbf{J}}
\newcommand{\Kv}{\mathbf{K}}
\newcommand{\Lv}{\mathbf{L}}
\newcommand{\Mv}{\mathbf{M}}
\newcommand{\Nv}{\mathbf{N}}
\newcommand{\Ov}{\mathbf{O}}
\newcommand{\Pv}{\mathbf{P}}
\newcommand{\Qv}{\mathbf{Q}}
\newcommand{\Rv}{\mathbf{R}}
\newcommand{\Sv}{\mathbf{S}}
\newcommand{\Tv}{\mathbf{T}}
\newcommand{\Uv}{\mathbf{U}}
\newcommand{\Vv}{\mathbf{V}}
\newcommand{\Wv}{\mathbf{W}}
\newcommand{\Xv}{\mathbf{X}}
\newcommand{\Yv}{\mathbf{Y}}
\newcommand{\Zv}{\mathbf{Z}}

% bold greek lowercase
\newcommand{\alphav     }{\boldsymbol \alpha     }
\newcommand{\betav      }{\boldsymbol \beta      }
\newcommand{\gammav     }{\boldsymbol \gamma     }
\newcommand{\deltav     }{\boldsymbol \delta     }
\newcommand{\epsilonv   }{\boldsymbol \epsilon   }
\newcommand{\varepsilonv}{\boldsymbol \varepsilon}
\newcommand{\zetav      }{\boldsymbol \zeta      }
\newcommand{\etav       }{\boldsymbol \eta       }
\newcommand{\thetav     }{\boldsymbol \theta     }
\newcommand{\varthetav  }{\boldsymbol \vartheta  }
\newcommand{\iotav      }{\boldsymbol \iota      }
\newcommand{\kappav     }{\boldsymbol \kappa     }
\newcommand{\varkappav  }{\boldsymbol \varkappa  }
\newcommand{\lambdav    }{\boldsymbol \lambda    }
\newcommand{\muv        }{\boldsymbol \mu        }
\newcommand{\nuv        }{\boldsymbol \nu        }
\newcommand{\xiv        }{\boldsymbol \xi        }
\newcommand{\omicronv   }{\boldsymbol \omicron   }
\newcommand{\piv        }{\boldsymbol \pi        }
\newcommand{\varpiv     }{\boldsymbol \varpi     }
\newcommand{\rhov       }{\boldsymbol \rho       }
\newcommand{\varrhov    }{\boldsymbol \varrho    }
\newcommand{\sigmav     }{\boldsymbol \sigma     }
\newcommand{\varsigmav  }{\boldsymbol \varsigma  }
\newcommand{\tauv       }{\boldsymbol \tau       }
\newcommand{\upsilonv   }{\boldsymbol \upsilon   }
\newcommand{\phiv       }{\boldsymbol \phi       }
\newcommand{\varphiv    }{\boldsymbol \varphi    }
\newcommand{\chiv       }{\boldsymbol \chi       }
\newcommand{\psiv       }{\boldsymbol \psi       }
\newcommand{\omegav     }{\boldsymbol \omega     }

% bold greek uppercase
\newcommand{\Gammav     }{\boldsymbol \Gamma     }
\newcommand{\Deltav     }{\boldsymbol \Delta     }
\newcommand{\Thetav     }{\boldsymbol \Theta     }
\newcommand{\Lambdav    }{\boldsymbol \Lambda    }
\newcommand{\Xiv        }{\boldsymbol \Xi        }
\newcommand{\Piv        }{\boldsymbol \Pi        }
\newcommand{\Sigmav     }{\boldsymbol \Sigma     }
\newcommand{\Upsilonv   }{\boldsymbol \Upsilon   }
\newcommand{\Phiv       }{\boldsymbol \Phi       }
\newcommand{\Psiv       }{\boldsymbol \Psi       }
\newcommand{\Omegav     }{\boldsymbol \Omega     }


% Abhi messing around with examdoc
\qformat{\textbf{{\Large \thequestion \; \; \thequestiontitle \ (\totalpoints \ points)}} \hfill}
\renewcommand{\thequestion}{\arabic{question}}
\renewcommand{\questionlabel}{\thequestion.}

\renewcommand{\thepartno}{\arabic{partno}}
\renewcommand{\partlabel}{\thepartno.}
\renewcommand{\partshook}{\setlength{\leftmargin}{0pt}}

\renewcommand{\thesubpart}{\alph{subpart}}
\renewcommand{\subpartlabel}{(\thesubpart)}

\renewcommand{\thesubsubpart}{\roman{subsubpart}}
\renewcommand{\subsubpartlabel}{\thesubsubpart.}

% copied from stack overflow, as all good things are
\newcommand\invisiblesection[1]{%
  \refstepcounter{section}%
  \addcontentsline{toc}{section}{\protect\numberline{\thesection}#1}%
  \sectionmark{#1}}

% quite possibly the worst workaround i have made for this class
\newcommand{\sectionquestion}[1]{
\titledquestion{#1}
\invisiblesection{#1}
~\vspace{-1em}
}

% hack for question numbers in table
\usepackage{regexpatch}
\makeatletter
\xpatchcmd*\@multicolumntable{|c|c}{|l|c}{}{}
\xpatchcmd\questions{\def\@currentlabel{\thequestiontitle}}{\def\@currentlabel{\thequestion. \thequestiontitle}}{}{}
\makeatother


\begin{document}

\section*{Instructions for Specific Problem Types}

For ``Select One" questions, please fill in the appropriate bubble completely:

\begin{quote}
\textbf{Select One:} Who taught this course?
\begin{list}{}
     \item\CIRCLE{} Matt Gormley
     \item\Circle{} Marie Curie
     \item\Circle{} Noam Chomsky
\end{list}
\end{quote}

If you need to change your answer, you may cross out the previous answer and bubble in the new answer:

\begin{quote}
\textbf{Select One:} Who taught this course?
\begin{list}{}
     \item\CIRCLE{} Matt Gormley
     \item\Circle{} Marie Curie\\
     \xcancel{\CIRCLE}{} Henry Chai
\end{list}
\end{quote}


For ``Select all that apply" questions, please fill in all appropriate squares completely:

\begin{quote}
\textbf{Select all that apply:} Which are scientists?
    \begin{list}{}
    \item $\blacksquare$ Stephen Hawking 
    \item $\blacksquare$ Albert Einstein
    \item $\blacksquare$ Isaac Newton
    \item $\square$ I don't know
\end{list}
\end{quote}

Again, if you need to change your answer, you may cross out the previous answer(s) and bubble in the new answer(s):

\begin{quote}
\textbf{Select all that apply:} Which are scientists?
    \begin{list}{}
    \item $\blacksquare$ Stephen Hawking 
    \item $\blacksquare$ Albert Einstein
    \item $\blacksquare$ Isaac Newton\\
    \xcancel{$\blacksquare$} I don't know
\end{list}
\end{quote}

For questions where you must fill in a blank, please make sure your final answer is fully included in the given space. You may cross out answers or parts of answers, but the final answer must still be within the given space.

\begin{quote}
\textbf{Fill in the blank:} What is the course number?

\begin{tcolorbox}[fit,height=1cm, width=4cm, blank, borderline={1pt}{-2pt},nobeforeafter]
    \begin{center}\huge10-601\end{center}
    \end{tcolorbox}\hspace{2cm}
    \begin{tcolorbox}[fit,height=1cm, width=4cm, blank, borderline={1pt}{-2pt},nobeforeafter]
    \begin{center}\huge10-\xcancel{6}301\end{center}
    \end{tcolorbox}
\end{quote}


% \section*{Instructions for Specific Problem Types}

For ``Select One" questions, please fill in the appropriate bubble completely:

\begin{quote}
\textbf{Select One:} Who taught this course?
    \begin{checkboxes}
     \CorrectChoice Matt Gormley
     \choice Marie Curie
     \choice Noam Chomsky
    \end{checkboxes}
\end{quote}

If you need to change your answer, you may cross out the previous answer and bubble in the new answer:

\begin{quote}
\textbf{Select One:} Who taught this course?
    {
    \begin{checkboxes}
     \CorrectChoice Henry Chai
     \choice Marie Curie \checkboxchar{\xcancel{\blackcircle}{}}
     \choice Noam Chomsky
    \end{checkboxes}
    }
\end{quote}

For ``Select all that apply" questions, please fill in all appropriate squares completely:

\begin{quote}
\textbf{Select all that apply:} Which are scientists?
    \begin{checkboxessquare}
    \CorrectChoice Stephen Hawking 
    \CorrectChoice Albert Einstein
    \CorrectChoice Isaac Newton
    \choice I don't know
    \end{checkboxessquare}
\end{quote}

Again, if you need to change your answer, you may cross out the previous answer(s) and bubble in the new answer(s):

\begin{quote}
\textbf{Select all that apply:} Which are scientists?
    \begin{checkboxessquare}
    \CorrectChoice Stephen Hawking 
    \CorrectChoice Albert Einstein
    \CorrectChoice Isaac Newton
    \choice I don't know
    \end{checkboxessquare}
\end{quote}

For questions where you must fill in a blank, please make sure your final answer is fully included in the given space. You may cross out answers or parts of answers, but the final answer must still be within the given space.

\begin{quote}
\textbf{Fill in the blank:} What is the course number?

\begin{tcolorbox}[fit,height=1cm, width=4cm, blank, borderline={1pt}{-2pt},nobeforeafter]
    \begin{center}\huge10-425\end{center}
    \end{tcolorbox}\hspace{2cm}
    \begin{tcolorbox}[fit,height=1cm, width=4cm, blank, borderline={1pt}{-2pt},nobeforeafter]
    \begin{center}\huge10-\xcancel{4}625\end{center}
    \end{tcolorbox}
\end{quote}

% \clearpage

\begin{questions}

\sectionquestion{Deep Models for Vision}
\begin{parts}
% TODO
% \part[1] In a BERT model, which is an encoder-only transformer network, the [CLS] token is used to aggregate the information from the entire sequence into a single vector, which can be used for many downstream tasks such as emotion detection, etc.\\
% \textbf{True or False:} The use of the [CLS] token at the very end of the input sequence will lead to a better performing model than if the [CLS] token was used at the beginning of the input sequence.
% \begin{checkboxes}
%     \choice True 
%     \choice False
% \end{checkboxes}
% \begin{soln}
%     False
% \end{soln}
% \begin{qauthor} 
%     Aravind, Lecture 5
% \end{qauthor}

\part[1] In a Vision Transformer (ViT) model, which is an encoder-only transformer network, the extra learnable [class] embedding is analogous to the [CLS] token in a BERT model. It is used to aggregate the information from the entire image into a single vector, which can be used for many downstream tasks such as object detection, etc.\\
\textbf{True or False:} The use of the extra learnable [class] embedding at the beginning of the input sequence, in principle, will lead to a better-performing model than if the extra learnable [class] embedding was used at the very end of the input sequence.
\begin{checkboxes}
    \choice True 
    \choice False
\end{checkboxes}
\begin{soln}
    False
\end{soln}
\begin{qauthor} 
    Aravind, Lecture 5
\end{qauthor}

\part[2] 
\textbf{Select One:} We have the following image and want to apply the following $3 \times 3$ kernel, with a padding of $1$ and stride of $2$. After applying this kernel, what is the value at position $(0, 1)$ (0-indexed)?

\begin{figure*}[h]
\begin{center}
$\begin{bmatrix}
1 & 2 & 2 & 1 & 1\\
2 & 4 & 4 & 2 & 3\\
1 & 3 & 3 & 1 & 4\\
1 & 2 & 3 & 1 & 3\\
0 & 1 & 4 & 2 & 0
\end{bmatrix}$
\caption{Image Matrix}
\end{center}
\end{figure*}

\begin{figure*}[h]
\begin{center}
$\begin{bmatrix}
0 & 1 & 0 \\
1 & -4 & 1 \\
0 & 1 & 0
\end{bmatrix}$
\caption{Kernel Matrix}
\end{center}
\end{figure*}

\begin{checkboxes}
    \choice 1 
    \choice 2
    \choice -1
    \choice 0
\end{checkboxes}
\begin{soln}
    -1\\
$\begin{bmatrix}
0 & 0 & 0\\
2 & 2 & 1\\
4 & 4 & 2\\
\end{bmatrix} * \begin{bmatrix}
0 & 1 & 0 \\
1 & -4 & 1 \\
0 & 1 & 0
\end{bmatrix} = 2 + 2(-4) + 1 + 4 = -1$\\
\end{soln}
\begin{qauthor} 
    Shreeya, Lecture 5
\end{qauthor}
\end{parts}


\clearpage
\sectionquestion{GANs}
\begin{parts}
% TODO 
\part[1] \textbf{True or False:} In the training process of a Generative Adversarial Network (GAN), if the input noise vector \( z \) to the generator \( G_{\theta} \) is completely random and lacks any discernible pattern, it becomes impossible to perform backpropagation through the generator \( G_{\theta} \).
\begin{checkboxes}
    \choice True 
    \choice False
\end{checkboxes}
\begin{soln}
    False
\end{soln}
\begin{qauthor} 
    Aravind, Lecture 6
\end{qauthor}

\part[1] \textbf{Select One: }Which of the following statements best describes the role of the discriminator in a Generative Adversarial Network (GAN)?
\begin{checkboxes}
    \choice The discriminator generates new data samples from the latent space. 
    \choice The discriminator updates the generator’s parameters to improve the quality of generated samples.
    \choice The discriminator distinguishes between real and generated data samples, providing feedback to the generator.
    \choice The discriminator minimizes the difference between real and generated data distributions.
\end{checkboxes}
\begin{soln}
    C
\end{soln}
\begin{qauthor} 
    Shreeya, Lecture 6
\end{qauthor}



\end{parts}

\clearpage
\sectionquestion{VAEs}
\begin{parts}
% TODO 
\part[0] \textbf{Select all that apply:} Which of the following are true statements about KL divergence?
\begin{checkboxessquare}
     \choice KL divergence is symmetric: $\text{KL}(p || q) = \text{KL}(q || p)$.
     \choice KL$(p||q)$ is minimized when distributions $p, q$
     are equivalent, i.e. $q(x) = p(x)$ for all $x \in X$.
     \choice Maximizing ELBO is the same as minimizing KL divergence. 
     \choice $0 \leq \text{KL}(p || q) \leq 1$ for all probability distributions $p$ and $q$. 
     \choice None of the above.
\end{checkboxessquare}
\begin{soln}
    B, C
\end{soln}
\begin{qauthor} 
    Katie, Lecture 7
\end{qauthor}
\part[1]
\textbf{True or False:} In Variational Autoencoders (VAEs), we sample from the latent distribution to generate the output. Since sampling prevents backpropagation through the network, neural networks cannot be used to build VAEs.
\begin{checkboxes}
    \choice True 
    \choice False
\end{checkboxes}
\begin{soln}
    False
\end{soln}
\begin{qauthor} 
    Shrikara, Lecture 7
\end{qauthor}
\end{parts}

\clearpage
\sectionquestion{Diffusion Models}
\begin{parts}
% TODO 
\part[1]
\textbf{Select all that apply:}Why must the noise scaling factor $\alpha_t$ in Diffusion Probabilistic Models (DDPMs) follow a schedule?
\begin{checkboxessquare}
    \choice To ensure smooth addition of noise, preventing excessive corruption of data early on.
    \choice To allow the model to gradually learn representations at different noise levels.
    \choice To add noise in a completely random fashion, making the model more robust.
    \choice To ensure the reverse process can recover the original data distribution effectively.
    \choice To control the amount of noise added during each step, ensuring balanced corruption across all timesteps
\end{checkboxessquare}
\begin{soln}
    A, D
\end{soln}
\begin{qauthor} 
    Shrikara, Lecture 8
\end{qauthor}

\part[1] \textbf{Select One:} Suppose you are training a neural network $p(x)$ to approximate a known but hard-to-define probability distribution $q(x)$. We choose as our objective function $KL(q \parallel p) = \mathbb{E}_q  \left[ \log \frac{q(x)}{p(x)}\right]$. Which of the following describes how to select the optimal p(x) to minimize this objective function?
\begin{checkboxes}
    \choice Select p(x) to be as large as possible in order to minimize the fraction $\frac{q(x)}{p(x)}$.
    \choice Select p(x) to be as small as possible in order to maximize the fraction $\frac{q(x)}{p(x)}$.
    \choice Select p(x) to be as close as possible to q(x) in order to make the fraction $\frac{q(x)}{p(x)}$ equal 1.
    \choice Select p(x) to be the uniform distribution so this term is independent of the parameters of the network and can be ignored.
\end{checkboxes}
\begin{soln}
    C, log(1) = 0 and the KL divergence is non-negative. Therefore 0 is an optimal value.
\end{soln}
\begin{qauthor} 
    Jacob, Lecture 6
\end{qauthor}


\part[1] \textbf{Select One:} Let function f(x, N) be the function which takes an image x and adds Gaussian noise to the image N times such that the output is the Nth step of forward diffusion in a DDPM model. What is the time complexity of an optimal implementation of this function? 
\begin{checkboxes}
    \choice O(1)
    \choice O(N)
    \choice O(h*w) where h and w are the image height and width
    \choice O(N*h*w) where h and w are the image height and width
    
\end{checkboxes}
\begin{soln}
    A. Adding Gaussian noise N times can be mathematically equivalently implemented by adding lots of Gaussian noise just once. 
\end{soln}
\begin{qauthor} 
    Jacob, Homework 2 and Lecture 8
\end{qauthor}

\end{parts}

\end{questions}

% Commented for quizzes
%\clearpage
\begin{center}
Do not remove this page! Use this page for scratch work.
\end{center}
\clearpage
\begin{center}
Do not remove this page! Use this page for scratch work.
\end{center}
\clearpage
\begin{center}
Do not remove this page! Use this page for scratch work.
\end{center}
\clearpage
\begin{center}
Do not remove this page! Use this page for scratch work.
\end{center}

\end{document}