\sectionquestion{Recommender Systems}

\begin{parts}

\part Neural the Narwhal asks his friends (6 users) to rate their favorite sea animals (8 items). He has a sparse ratings matrix of size $6 \times 8$, and he wants to use collaborative filtering to factor this matrix into a user matrix and an item matrix.

\begin{subparts}

    \subpart[2] \textbf{Select All That Apply:}  He thinks, since each latent feature captures information about the underlying data, he wants to factor the user matrix $\Uv$ into size $6 \times 10000$ and the item matrix $\Vv$ into size $8 \times 10000$. Is Neural's choice of $10000$ latent features a reasonable decision?
    {%
    \checkboxchar{$\Box$} \checkedchar{$\blacksquare$} % change checkbox style locally
    \begin{checkboxes}
     \choice Yes, because more latent features are able to capture more data about the factors that go into the ratings, thus leading to model that generalizes better.
     \choice Yes, because after these matrices are learned, predictions will be made on new users, causing the user matrix to grow.
     \choice No, because the matrices that are learned may overfit on the dataset.
     \choice No, because the matrix cannot be factored at all unless the number of latent features is $\min(6,8)$.
     \choice None of the above
    \end{checkboxes}
    }
    \begin{soln}
    C
    \end{soln}
    \begin{qauthor}
    Emily (edited by Matt), Recommender Systems\\
    AFTER FEEDBACK: took by taking out line saying that Neural is wrong and reword question to not imply if this is a good or bad method. changed short answer into select all that apply options.
    \end{qauthor}

    \uplevel{Neural now tries using only 2 features in the latent space, so the user matrix $\Uv$ is of size $6 \times 2$ and so the item matrix $\Vv$ is of size $8 \times 2$.
    Suppose an oracle provides a new user vector $\uv_i \in \Rb^2$, i.e. with 2 latent features. }

    \subpart[1] \textbf{Mathematical answer:} 
    Write an expression to predict the rating given to the $j$ item (sea animal) by the new user vector $\uv_i$.
    \begin{tcolorbox}[fit,height=2cm, width=7cm, blank, borderline={1pt}{-2pt}]
    %solution
    \end{tcolorbox}
    \begin{soln}
    $\uv_i^T \Vv_{j,\cdot}$
    \end{soln}
    \begin{qauthor}
    Matt
    \end{qauthor}
    
    \subpart[2] \textbf{Mathematical answer:} 
    Write an expression to predict \textit{which} item (sea animal) the user will rate the highest; this integer is the recommendation to the user.
    (Hint: you may use the $\argmax$ function to obtain the index of the maximum value in a vector or matrix.)
    \begin{tcolorbox}[fit,height=2cm, width=7cm, blank, borderline={1pt}{-2pt}]
    %solution
    \end{tcolorbox}
    \begin{soln}
    $\texttt{argmax}(\uv_i V^T)$
    \end{soln}
    \begin{qauthor}
    Emily, Recommender Systems
    \end{qauthor}
    \begin{qtester}
    Somewhat open-ended with how Neural is wrong in part a -- students may end up saying ``predictions aren't guaranteed to be good even with all important features'' or something similar. The goal of the question seems to be getting at matrix factorization as an approximation/a low dimensional projection of users and items. The matrices Neural could find could simply reconstruct the original matrix exactly, which may make the question's discussion of ``every important feature'' confusing.

    Part b looks fine.
    \end{qtester}
    
\end{subparts}

\end{parts}

