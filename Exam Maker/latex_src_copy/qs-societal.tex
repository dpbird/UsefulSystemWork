\sectionquestion{Societal Impacts of ML}

\begin{parts}

% Question 9
\part[1] \textbf{Select one:} Suppose we have a model that has a 0.5 error rate on each of two distinct groups in the dataset. Which of the following fairness metrics will \textbf{always} be satisfied? 

    % \checkboxchar{$\Box$} \checkedchar{$\blacksquare$}
    \begin{checkboxes}
        \choice False Negative Rate (FNR) parity
        \choice False Positive Rate (FPR) parity
        % \choice Negative Predictive Value (NPV) parity
        % \choice Positive Predictive Value (PPV) parity
        \choice Error parity 
        % \choice Statistical parity (or Selection rate parity)
        \choice A, B, and C
        \choice All of the above
        \choice None of the above
    \end{checkboxes}

    \begin{soln}
        (c) Error parity
    \end{soln}
    \begin{qauthor}
        Annie, fairness metrics and societal impacts
    \end{qauthor}

% Question 10
\part Markov is trying to build a model on predicting whether Cognitive Machine University (CMU') will admit a student. They are given the following dataset, split into two groups (Red/Blue), with entirely binary features.
    \begin{itemize}
        \item \textbf{GPA above 3.7:} 1 indicates that the student has a GPA above 3.7 reported on their application, 0 otherwise
        \item \textbf{Legacy:} takes on the value 1 if the student comes from a legacy family, 0 otherwise
        \item \textbf{Athelete:} 1 if student is recruited for sports, 0 otherwise
        \item \textbf{Admitted?:} shows the true label, 1 meaning the student was actually admitted, 0 otherwise
    \end{itemize}

    \begin{table}[htbp]
    \centering
    % \caption{Data Table}
    \label{tab:data}
    \begin{tabular}{|c|c|c|c|c|}
        \toprule
        Protected Attribute & GPA above 3.7 & Legacy & Athelete & Admitted? \\
        \midrule
        Red & 1 & 0 & 1 & 1 \\
        Red & 0 & 1 & 0 & 0 \\
        Red & 1 & 0 & 0 & 0 \\
        Red & 0 & 1 & 1 & 0 \\
        Red & 1 & 0 & 0 & 1 \\
        \midrule
        Blue & 0 & 0 & 1 & 0 \\
        Blue & 1 & 1 & 0 & 1 \\
        Blue & 0 & 0 & 0 & 1 \\
        Blue & 1 & 0 & 1 & 0 \\
        Blue & 0 & 1 & 0 & 0 \\
        \bottomrule
    \end{tabular}
\end{table}

Markov uses a model where we predict 1 if the ``majority" of the features for a student is 1 (i.e. at least 2 of the features has value 1 for that student), and 0 otherwise.

\clearpage
\begin{subparts}
    \subpart[1] \textbf{Numerical answer:} What is the negative predictive value on the Red group?

    \begin{tcolorbox}[fit,height=1cm, width=2cm, blank, borderline={1pt}{-2pt}]
    %solution
    \end{tcolorbox}
    \begin{soln}
    2/3
    \end{soln}

    \subpart[1] \textbf{Numerical answer:} What is the negative predictive value on the Blue group?
    \begin{tcolorbox}[fit,height=1cm, width=2cm, blank, borderline={1pt}{-2pt}]
    %solution
    \end{tcolorbox}
    \begin{soln}
    2/3
    \end{soln}

    \subpart[1] \textbf{True or False:} Does Markov's majority vote classifier achieve negative predictive value (NPV) parity?
    \begin{checkboxes}
     \choice True 
     \choice False
    \end{checkboxes}

    \begin{soln}
        True
    \end{soln}

    \subpart[1] \textbf{Numerical answer:} What is the false positive rate on the Red group?

    \begin{tcolorbox}[fit,height=1cm, width=2cm, blank, borderline={1pt}{-2pt}]
    %solution
    \end{tcolorbox}
    \begin{soln}
    1/3
    \end{soln}

    \subpart[1] \textbf{Numerical answer:} What is the false positive rate on the Blue group?
    \begin{tcolorbox}[fit,height=1cm, width=2cm, blank, borderline={1pt}{-2pt}]
    %solution
    \end{tcolorbox}
    \begin{soln}
    1/3
    \end{soln}

    \subpart[1] \textbf{True or False:} Does Markov's majority vote classifier achieve false positive rate (FPR) parity?
    \begin{checkboxes}
     \choice True 
     \choice False
    \end{checkboxes}

    \begin{soln}
        True
    \end{soln}

    \subpart[2] \textbf{Short answer:} Incorporating your responses from the previous parts, briefly discuss the fairness of the model. Can we say that the model is a fair between the Red and Blue groups?
    \fillwithlines{12em}
    \begin{soln}
    TODO: update this solution now that we compute NPV.
    
        Even though we achieve both false negative and false positive parity, it does not imply that this model is necessarily fair. There are other metrics we should test against (i.e. error parity, selection rate parity, etc.)
    \end{soln}

\clearpage
    \subpart[1] \textbf{Short answer:} What are some societal impacts of this model. In particular, what consequences might false negatives and false positive predictions have on the applicants?
    \fillwithlines{12em}
    \begin{soln}
    TODO: update this solution now that we compute NPV.
    
        A false negative mistake will take away a qualified student's acceptance to CMU, and a false positive mistake will admit non-qualified candidates to the university.
    \end{soln}
    
\end{subparts}

\begin{qauthor}
    Annie, societal impacts
\end{qauthor}



\end{parts}